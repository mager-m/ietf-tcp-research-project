\section{Introduction}
\label{sec:introduction}

The nature of modern computer systems with highly distributed and diversified infrastructure presents a need for standardized network management protocols and data modeling languages. With the introduction of the Network Configuration Protocol (NETCONF)~\cite{RFC6241} and the data modeling language YANG~\cite {RFC7950} by the IETF, this need has largely been satisfied and gained wide acceptance throughout the network equipment industry. Other industries are starting to take an interest in the standardized and proven technologies to handle management of complex systems in their domains. This greatly extends the scope of NETCONF/YANG from mainly network devices to a wide range of industries.

The growing acceptance and usage of NETCONF/YANG drives the need for common, well-tested and future-proof management frameworks that work across many environments. While there are several commercial options available, like \textit{ConfD} from Cisco~\cite{confd}, they are mostly closed-source and do not allow for any modifications or adjustments to the base product. Besides commercial options, multiple open-source management frameworks exist, allowing companies to gain full control and adjust the framework to their individual needs.

This paper introduces and evaluates the open source network management framework \textit{Clixon} and presents a comparison of its capabilities with the well-known open source framework \textit{Netopeer2}~\cite{netopeer2} and the commercial, closed-source alternative \textit{ConfD}. \textit{Clixon} is intended for network devices and other computer systems and provides support for a large feature set, such as datastores and appropriate transaction mechanisms. Developers can easily integrate in the framework by providing plugins to react to configuration changes or return state data. To interact with the management agent, the network management protocols NETCONF and RESTCONF as well as a rudimentary command line interface (CLI) are provided~\cite{clixon-documentation}.

The rest of this paper is organized as follows. Section~\ref{sec:clixon-framework} introduces \textit{Clixon}, presents an overview of the architecture and draws a comparison with other open-source frameworks. The \textit{ietf-tcpm-yang-tcp}~\cite{draft-ietf-tcpm-yang-tcp} YANG model is introduced in section~\ref{sec:prototype} and the corresponding \textit{Clixon} plugin is shown. In addition, the portability of the prototype and the underlying framework is analyzed. Section~\ref{sec:conclusion} draws a final conclusion.
